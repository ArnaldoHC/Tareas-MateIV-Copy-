\documentclass[12pt]{exam}
\usepackage[utf8]{inputenc}		% Caracteres latinos
\usepackage[spanish]{babel}		% Idioma español
\usepackage{geometry}			% Organizar el documento
\usepackage{graphicx}			% Incluir gráficos
\usepackage{makecell}			% Para personalizar las celdas de una tabla
\usepackage[nohdr]{mathexam}	% Añadimos el paquete mathexam (sin header)
\usepackage{amsmath}
\usepackage{amsfonts}
\usepackage{amssymb}
\usepackage{mathtools}
\usepackage{tikz,pgfplots}
\usepgfplotslibrary{polar}
\usepackage[shortlabels]{enumitem}
 \renewcommand{\baselinestretch}{1.5}
\usepackage{mathtools}
\usepackage{bm}
\usepackage{esvect}
\usepackage[fleqn]{mathtools}
\usepackage{relsize}
\usepackage{multirow}
\usepackage{multicol}
\usepackage[document]{ragged2e}
 \usepackage{textpos}
\usepackage{tcolorbox}
\usepackage{hyperref}
\usepackage{mathdesign}

% Definimos la geometría de la primera página
\geometry{
	a4paper,                    % Tamaño del documento
	hmargin = {1.7cm, 1.6cm}, 	% Margen horizontal izquierdo, derecho
	vmargin = {1cm, 1cm},	    % Margen vertical superior, inferior
	headsep = 4mm,				% Separación entre el encabezado y el texto
	head = .2cm,				% Tamaño del encabezado
	% marginparsep = 5mm, 		% Seperación entre las notas y el texto
	% marginpar = 1.5cm,		% Tamaño de las notas
	includeall,                 % incluye el encabezado, footer y notas dentro del tamaño del documento
	nomarginpar,	            % Elimina las notas
	foot = 1cm,                 % Tamaño del footer
	twoside,                	% Habilita el modo de impresión a doble cara
}

\selectlanguage{spanish}        % Selecciona el idioma
\spanishdecimal{.}


% DEFINIMOS EL PIE
%\rfoot{Página \thepage\ de \numpages}
\newcommand{\iuni}{\pmb{\hat{\imath}}}
\newcommand{\juni}{\pmb{\hat{\jmath}}}
\newcommand{\kuni}{\pmb{\hat{k}}}
\renewcommand{\sin}{\,\text{sen}\,}
\newcommand*\colvec[3][]{
    \begin{pmatrix}\ifx\relax#1\relax\else#1\\\fi#2\\#3\end{pmatrix}
}
%\colvec{a}{b} para dos 
% \colvec[a]{b}{c} para tres
% DOCUMENTO
\begin{document}

\centering


\Large 
\textbf{\huge Reposición 3 \\ \large }

\small
\vskip10pt
\normalsize

\pointpoints{punto}{puntos}
\pointformat{\bfseries\boldmath(\thepoints)}
\vskip10pt

    
    \begin{questions}
    
     \question% 
     Para los siguientes sistemas
     \begin{enumerate}[i.]
         \item Calcula los eigenvalores
         \item Para cada eigenvalor, determina los eigenvectores asociados.
         \item Esboza el campo de direcciones y el retrato fase para el sistema.
         \item Encuentra la solución general y particular con el valor inicial dado.
         \item ¿Qué tipo de punto de equilibrio es el origen?
         \item Dibuja las gráficas $x(t)$ y $y(t)$ de la solución con la condición inicial dada.
     \end{enumerate}
    \vskip 20pt
    
    \begin{enumerate}[a)]
        \item $\dfrac{d\mathbf{Y}}{dt}=\colvec{-2\quad-1}{\phantom{-}1\quad-4}\mathbf{Y}$ con la condición inicial $\mathbf{Y}_0=(1,0)$
        \item $\dfrac{d\mathbf{Y}}{dt}=\colvec{-1\quad\phantom{-}2}{-1\quad-1}\mathbf{Y}$, con condición inicial $\mathbf{Y}_0=(0,1)$
        \item $\dfrac{d\mathbf{Y}}{dt}=\colvec{\;3\quad4\;\;}{\;1\quad0\;\;}\mathbf{Y}$, con condición inicial $\mathbf{Y}_0=(0,-1)$
    \end{enumerate}

    
     \question% 
     Para el sistema $$\frac{dx}{dt}=2x\atop \frac{dy}{dt}=y$$
     afirmamos que la curva $\mathbf{Y}(t)=(e^{2t},3e^t)$ es una solución. Su posición inicial es $\mathbf{Y}(0)=(1,3)$.
     \begin{enumerate}[a)]
         \item Verifica que $\mathbf{Y}(t)=(e^{2t},3e^t)$ es una solución.
         \item Usa el método de Euler con tamaño de paso $\Delta t=0.5$ para aproximar esta solución y determina su cercanía a la solución real cuando $t=2$, $t=4$ y $t=6$.
         \item Ahora usa el método de Euler con tamaño de paso $\Delta t=0.1$ para hacer la aproximación. Calcula que tan cerca está de la solución real cuando $t=2$, $t=4$ y $t=6$.
     \end{enumerate}
     Apóyate en una calculadora o computadora para realizar los cálculos y gráficas. No olvides poner tu codigo.

     \question
     Los incisos a - f se refieren a los siguientes sistemas de ecuaciones:
     
     $
     \begin{array}{lrclclrcl}
        i)&\frac{dx}{dt}& = & 10x\left(1-\frac{x}{10}\right)-20xy& &ii)& \frac{dx}{dt} & = & 0.3x-\left(\frac{xy}{100}\right)\\
        & \frac{dy}{dt} & = & -5y+\frac{xy}{20} & & &\frac{dy}{dt}& = & 15y\left(1-\frac{y}{15}\right)+25xy
     \end{array}
     $
     \begin{enumerate}[a)]
         \item En uno de esos sistemas, las presas son animales muy grandes y los depredadores son animales muy pequeños, tales como elefantes y mosquitos. Se requieren entonces muchos de éstos para cazar una presa, pero cada animal devorado proporciona gran beneficio para la población depredadora. El otro sistema tiene depredadores muy grandes y presas muy pequeñas. Identifica cada sistema y da una justificación para tu respuesta.
         \item Encuentra todos los puntos de equilibrio para los dos sistemas. Explica la importancia de esos puntos en términos de las poblaciones presa y depredadora.
         \item Para cada sistema, describe el comportamiento de la población depredadora si las presas están extintas. Tomando en cuenta la ausencia de presas, bosqueja la línea fase y las gráficas de la población depredadora como función del tiempo para varias soluciones. Interpreta luego esas gráficas para la población depredadora.
     \end{enumerate}

     \question% 
     Convierte la ecuación diferencial de tercer orden $$\frac{d^3y}{dt^3}+p\frac{d^2y}{dt^2}+q\frac{dy}{dt}+ry=0$$ donde $p$, $q$ y $r$ son constantes, a un sistema lineal tridimensional escrito en forma matricial.

     \question% 
     Considera el sistema parcialmente acoplado $$\begin{array}{rcl}
          \frac{dx}{dt}&=&xy  \\
          \frac{dy}{dt}&=&y+1 
     \end{array}$$
     \begin{enumerate}[a)]
         \item Obtén la solución general.
         \item Calcula los puntos de equilibrio del sistema.
         \item Encuentra la solución que satisface la condición inicial $(x_0,y_0)=(1,0)$.
     \end{enumerate}



    \end{questions}

\end{document}
