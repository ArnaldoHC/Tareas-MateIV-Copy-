\documentclass[12pt]{exam}
\usepackage[utf8]{inputenc}		% Caracteres latinos
\usepackage[spanish]{babel}		% Idioma español
\usepackage{geometry}			% Organizar el documento
\usepackage{graphicx}			% Incluir gráficos
\usepackage{makecell}			% Para personalizar las celdas de una tabla
\usepackage[nohdr]{mathexam}	% Añadimos el paquete mathexam (sin header)
\usepackage{amsmath}
\usepackage{amsfonts}
\usepackage{amssymb}
\usepackage{mathtools}
\usepackage{tikz,pgfplots}
\usepgfplotslibrary{polar}
\usepackage[shortlabels]{enumitem}
 \renewcommand{\baselinestretch}{1.5}
\usepackage{mathtools}
\usepackage{bm}
\usepackage{esvect}
\usepackage[fleqn]{mathtools}
\usepackage{relsize}
\usepackage{multirow}
\usepackage{multicol}
\usepackage[document]{ragged2e}
 \usepackage{textpos}
\usepackage{tcolorbox}
\usepackage{hyperref}
\usepackage{mathdesign}

%\usepackage[]{mathptmx}        % A free version o Times Roman with mathematical symbols
%\usepackage{pzc}               % fuente cursiva (conjuntos) Zapf Chancery
%\usepackage{showframe}
%\usepackage{lipsum}

% DOCUMENTACIÓN DE LA CLASE EXAM
% http://ftp.inf.utfsm.cl/pub/tex-archive/macros/latex/contrib/exam/examdoc.pdf
% DOCUMENTACIÓN DE LA CLASE MATHEXAM
% http://ctan.dcc.uchile.cl/macros/latex/contrib/mathexam/doc/mathexam.pdf

% Definimos la geometría de la primera página
\geometry{
	a4paper,                    % Tamaño del documento
	hmargin = {1.7cm, 1.6cm}, 	% Margen horizontal izquierdo, derecho
	vmargin = {1cm, 1cm},	    % Margen vertical superior, inferior
	headsep = 4mm,				% Separación entre el encabezado y el texto
	head = .2cm,				% Tamaño del encabezado
	% marginparsep = 5mm, 		% Seperación entre las notas y el texto
	% marginpar = 1.5cm,		% Tamaño de las notas
	includeall,                 % incluye el encabezado, footer y notas dentro del tamaño del documento
	nomarginpar,	            % Elimina las notas
	foot = 1cm,                 % Tamaño del footer
	twoside,                	% Habilita el modo de impresión a doble cara
}

\selectlanguage{spanish}        % Selecciona el idioma
\spanishdecimal{.}

%\pagestyle{headandfoot}         % Nuestro examen tendrá encabezado y pié

% DEFINIMOS EL ENCABEZADO
%\header{
%\begin{tabular}{l c c c l}
%            \makecell{\includegraphics[height=2.5cm]{logo.png}} &
%            \makecell{\textbf{IPEA 215} \\Raúl Scalabrini Ortiz} &
%            \makecell{Examen} &
%            \makecell{Curso\\1er Año} &
%             \makecell[l]{Apellido y %Nombre:\enspace\makebox[2in]{\hrulefill}\\Fecha: \today}
%        \end{tabular}}{}{}

% DEFINIMOS EL PIE
%\rfoot{Página \thepage\ de \numpages}
\newcommand{\iuni}{\pmb{\hat{\imath}}}
\newcommand{\juni}{\pmb{\hat{\jmath}}}
\newcommand{\kuni}{\pmb{\hat{k}}}
\renewcommand{\sin}{\,\text{sen}\,}

% DOCUMENTO
\begin{document}

\centering


\Large 
\textbf{\huge Tarea 3\\ \large Otros tipos de ecuaciones. Métodos numéricos}

\small
Fecha de entrega viernes 5 de Noviembre
\vskip10pt

\normalsize

\pointpoints{punto}{puntos}
\pointformat{\bfseries\boldmath(\thepoints)}
\vskip10pt

    
    \begin{questions}
     % 1 %
     \question
     Resuelve las siguientes ecuaciones.
     \begin{enumerate} [a)]
			\item $(3x^2+y)dx+(x^2y-x)dy=0$
		  	\item $(2xy)dx+(y^2-3x^2)dy=0$
%           	\item $(y^2+2xy)dx-x^2dy=0$
     \end{enumerate}

     
     % 2 %
     \question
     Muestra que si $(\partial N/\partial x-\partial M/\partial y)/(xM-yN)$ solo depende del producto $xy$, es decir, $$\frac{\partial N/\partial x-\partial M/\partial y}{xM-yN}=H(xy),$$ entonces la ecuación $M(x,y)dx+N(x,y)dy=0$ tiene un factor integrante de la forma $\mu(xy)$. Proporciona la fórmula general para $\mu(xy)$.
     
     % 3 %
     \question
     Identifica la ecuación como homogénea, de Bernoulli, con coeficientes lineales o de la forma $y'=G(ax+by)$. Resuelve las ecuaciones.
     \begin{enumerate}[a)]
     	\item $(y-4x-1)^2dx-dy=0$
        \item $2tx\,dx+(t^2-x^2)dt=0$
        \item $(y^3-\theta y^2)d\theta+2\theta^2y\,dy=0$
        \item $\frac{dx}{dt}=\frac{x^2+t\sqrt{t^2+x^2}}{tx}$
        \item $\frac{dy}{dx}= \sin(x-y)$
        \item $\frac{dx}{dt}+tx^3+\frac{x}{t}=0$
        \item $\frac{dy}{dx}+y^3x+y=0$
        \item $(2x-y)dx+(4x+y-3)dy=0$
       \end{enumerate}
        
     % 4 %
     \question
     \begin{enumerate}[a)]
     		\item Muestra que la ecuación $dy/dx=f(x,y)$ es homogénea si y solo si $f(tx,ty)=f(x,y)$. Hint: usa $t=1/x$
            \item Una función $H(x,y)$ es homogénea de orden $\mathbf{n}$ si $H(tx,ty)=t^nf(x,y)$. Muestra que la ecuación $M(x,y)dx+N(x,y)dy=0$ es homogénea si $M(x,y)$ y $N(x,y)$ son homogéneas del mismo orden.
     \end{enumerate}
     

     % 5 %
     \question
     Una ecuacion de la forma $$\frac{dy}{dx}=P(x)y^2+Q(x)y+R(x)$$ se denomina ecuación de Riccat.
     \begin{enumerate}[a)]
     		\item Supóngase una solución $u(x)$. Muestra que la sustitución $y=u+1/v$ reduce la ecuación de Riccati a una ecuación lineal en $v$.
            \item Dado que $u(x)=x$ es una solución de $$\frac{dy}{dx}=x^3(y-x)^2+\frac{y}{x},$$ usa el resultado del inciso \textit{a)} para hallar todas las demás soluciones de esta ecuación.
     \end{enumerate}

     % 6 %
     \question
     Calcula la aproximación mediante el \textit{a)} el método de Euler y \textit{b)} el método de Euler mejorado de la solución $\phi(x)=e^{5x}$ de $y'=5y$ con $y(0)=1$. Completa las tablas.
    
    \small
     \begin{tabular}{cm{2.5cm}cc}
     \hline
     h&\centering{Aproximación de Euler}&Error&Error/h\\
     \hline
     1\\
     $10^{-1}$\\
     $10^{-2}$\\
     $10^{-3}$\\
     $10^{-4}$\\
     \hline
     \end{tabular}
     \begin{tabular}{cm{3cm}cc}
     \hline
     h&\centering{Aproximación de Euler mejorado}&Error&Error/$h^2$\\
     \hline
     1\\
     $10^{-1}$\\
     $10^{-2}$\\
     $10^{-3}$\\
     $10^{-4}$\\
     \hline
     \end{tabular}
     
  
     

     % 7 %
     \question
     Usa el método de Runge-Kutta de cuarto orden con $h=0.25$ para aproximar la solución del problema con valor inicial $$y'=2y-6,\quad y(0)=1$$ en $x=1$. Comprueba esta aproximación con la solución real $y=3-2e^{2x}$ evaluada en $x=1$.

        \end{questions}
        \vskip30pt


\end{document}
