\documentclass[12pt]{exam}
\usepackage[utf8]{inputenc}		% Caracteres latinos
\usepackage[spanish]{babel}		% Idioma español
\usepackage{geometry}			% Organizar el documento
\usepackage{graphicx}			% Incluir gráficos
\usepackage{makecell}			% Para personalizar las celdas de una tabla
\usepackage[nohdr]{mathexam}	% Añadimos el paquete mathexam (sin header)
\usepackage{amsmath}
\usepackage{amsfonts}
\usepackage{amssymb}
\usepackage{mathtools}
\usepackage{tikz,pgfplots}
\usepgfplotslibrary{polar}
\usepackage[shortlabels]{enumitem}
 \renewcommand{\baselinestretch}{1.5}
\usepackage{mathtools}
\usepackage{bm}
\usepackage{esvect}
\usepackage[fleqn]{mathtools}
\usepackage{relsize}
\usepackage{multirow}
\usepackage{multicol}
\usepackage[document]{ragged2e}
 \usepackage{textpos}
\usepackage{tcolorbox}
\usepackage{hyperref}
\usepackage{mathdesign}

\geometry{
	a4paper,                    % Tamaño del documento
	hmargin = {1.7cm, 1.6cm}, 	% Margen horizontal izquierdo, derecho
	vmargin = {1cm, 1cm},	    % Margen vertical superior, inferior
	headsep = 4mm,				% Separación entre el encabezado y el texto
	head = .2cm,				% Tamaño del encabezado
	% marginparsep = 5mm, 		% Seperación entre las notas y el texto
	% marginpar = 1.5cm,		% Tamaño de las notas
	includeall,                 % incluye el encabezado, footer y notas dentro del tamaño del documento
	nomarginpar,	            % Elimina las notas
	foot = 1cm,                 % Tamaño del footer
	twoside,                	% Habilita el modo de impresión a doble cara
}

\selectlanguage{spanish}        % Selecciona el idioma
\spanishdecimal{.}

%\pagestyle{headandfoot}         % Nuestro examen tendrá encabezado y pié

% DEFINIMOS EL ENCABEZADO
%\header{
%\begin{tabular}{l c c c l}
%            \makecell{\includegraphics[height=2.5cm]{logo.png}} &
%            \makecell{\textbf{IPEA 215} \\Raúl Scalabrini Ortiz} &
%            \makecell{Examen} &
%            \makecell{Curso\\1er Año} &
%             \makecell[l]{Apellido y %Nombre:\enspace\makebox[2in]{\hrulefill}\\Fecha: \today}
%        \end{tabular}}{}{}

% DEFINIMOS EL PIE
%\rfoot{Página \thepage\ de \numpages}
\newcommand{\iuni}{\pmb{\hat{\imath}}}
\newcommand{\juni}{\pmb{\hat{\jmath}}}
\newcommand{\kuni}{\pmb{\hat{k}}}
\renewcommand{\sin}{\,\text{sen}\,}
\newcommand*\colvec[3][]{
    \begin{pmatrix}\ifx\relax#1\relax\else#1\\\fi#2\\#3\end{pmatrix}
}
%\colvec{a}{b} para dos 
% \colvec[a]{b}{c} para tres
% DOCUMENTO
\begin{document}

\centering


\Large 
\textbf{\huge Tarea 7 \\ \large Sistemas Lineales}

\small
Fecha de entrega Viernes 14 de Enero
\vskip10pt
\normalsize

\pointpoints{punto}{puntos}
\pointformat{\bfseries\boldmath(\thepoints)}
\vskip10pt

    
    \begin{questions}
     % 1 % 
     \question
     Considera el sistema lineal correspondiente a la ecuación de segundo orden $$\frac{d^2y}{dt^2}+p\frac{dy}{dt}+qy=0$$
     \begin{enumerate}[a)]
         \item Si $q=0$ y $p\neq0$, encuentra todos los puntos de equilibrio.
         \item Si $q=p=0$, encuentra todos los puntos de equilibrio.
     \end{enumerate}


     % 2 
     \question% 
     Convierte la ecuación diferencial de tercer orden $$\frac{d^3y}{dt^3}+p\frac{d^2y}{dt^2}+q\frac{dy}{dt}+ry=0$$ donde $p$, $q$ y $r$ son constantes, a un sistema lineal tridimensional escrito en forma matricial.

     
     % 3 
     \question% 
     Para los siguientes sistemas lineales de la forma $\dfrac{d\mathbf{Y}}{dt}=\mathbf{AY}$ se especifica la matriz de coeficientes,  dos funciones y un valor inicial. 
     \begin{enumerate}[i)]
         \item      Verifica que las dos funciones son soluciones del sistema.
         \item		Si el inciso anterior se cumple, comprueba que las dos soluciones son linealmente independientes.
          \item		Si el inciso anterior se cumple, encuentra la solución del sistema lineal con el valor inicial dado.
    \end{enumerate}
\vskip 20pt
    \begin{enumerate}[a)]
        \item $\mathbf{A}=\left(\begin{matrix}-2&-1\\2&-5\end{matrix}\right)$
        
        Funciones: $\begin{matrix}\mathbf{Y}_1(t)=(e^{-3t}-2e^{-4t},e^{-3t}-4e^{-4t})\phantom{1}\\\mathbf{Y}_2(t)=(2e^{-3t}+e^{-4t},2e^{-3t}+2e^{-4t})\end{matrix}$
        
        Valor inicial: $\mathbf{Y}(0)=(2,3)$
        \item $\mathbf{A}=\left(\begin{matrix}2&3\\1&0\end{matrix}\right)$
        
        Funciones: $\begin{matrix}\mathbf{Y}_1(t)=(-e^{-t}+12e^{3t},e^{-t}+4e^{3t})\\\mathbf{Y}_2(t)=(-e^{-t},2e^{-t})\phantom{------}\end{matrix}$
        
        Valor inicial: $\mathbf{Y}(0)=(2,3)$
    \end{enumerate}

     % 4 
     \question% 
     Para los siguientes sistemas
     \begin{enumerate}[i)]
         \item Calcula los eigenvalores
         \item Para cada eigenvalor, determina los eigenvectores asociados.
         \item Esboza el campo de direcciones para el sistema, las soluciones de la línea recta y el retrato fase.
         \item Para cada eigenvalor, especifica la solución de línea recta que corresponde y traza sus gráficas $x(t)$ y $y(t)$.
         \item Determina la solución general.
         \item ¿Qué tipo de punto de equilibrio es el origen?
     \end{enumerate}
\vskip 20pt
    
    \begin{enumerate}[a)]
        \item $\colvec{\frac{dx}{dt}}{\frac{dy}{dt}}=\colvec{\;3\quad4\;\;}{\;1\quad0\;\;}\colvec{x}{y}$
        \item $\colvec{\frac{dx}{dt}}{\frac{dy}{dt}}=\colvec{\;4\quad2\;\;}{\;1\quad3\;\;}\colvec{x}{y}$
        \item $\dfrac{d\mathbf{Y}}{dt}=\colvec{\;3\quad\phantom{-}2\;\;}{\;0\quad-2\;\;}\mathbf{Y}$
        
    \end{enumerate}

     % 5 
     \question%
     Los siguientes sistemas lineales tienen eigenvalores complejos. Para cada sistema:
     \begin{enumerate}[i)]
         \item Encuentra los eigenvalores.
         \item Determina si el origen es un sumidero espiral, una fuente espiral o un centro.
         \item Encuentra la solución general y particular con el valor incial dado.
         \item Determina la dirección de las oscilaciones en el plano fase. ¿Van las soluciones en el sentido de las manecillas del reloj o al contrario alrededor del origen?
         \item Esboza el plano fase $xy$ y las gráficas $x(t)$ y $y(t)$ para las soluciones con las condiciones iniciales indicadas.
     \end{enumerate}
     \vskip20pt
     \begin{enumerate}[a)]
         \item $\dfrac{d\mathbf{Y}}{dt}=\colvec{2\quad-6}{2\quad\phantom{-}1}\mathbf{Y}$, con condición inicial $\mathbf{Y}_0=(2,1)$
         \item $\dfrac{d\mathbf{Y}}{dt}=\colvec{-1\quad\phantom{-}2}{-1\quad-1}\mathbf{Y}$, con condición inicial $\mathbf{Y}_0=(0,1)$
         \item $\dfrac{d\mathbf{Y}}{dt}=\colvec{\phantom{-}0\quad2}{-2\quad0}\mathbf{Y}$, con condición inicial $\mathbf{Y}_0=(1,0)$
     \end{enumerate}



     % 6 
     \question% 
     Para la ecuación de segundo orden $$\frac{d^2y}{dt^2}+p\frac{dy}{dt}+qy=0$$
     \begin{enumerate}[a)]
         \item Escribe esta ecuación como un sistema lineal de primer orden.
         \item ¿Qué condiciones en $p$ y $q$ garantizan que los eigenvalores del correspondiente sistema lineal son complejos?
         \item ¿Qué relación entre $p$ y $q$ garantiza que el origen es un sumidero espiral? ¿Qué relación asegura que es una fuente espiral? ¿Y cuál otra nos dice que el origen es un centro?
         \item Si los eigenvalores son complejos, ¿qué condiciones en $p$ y $q$ nos indican que las soluciones se mueven en espiral alrededor del origen en el sentido de las manecillas del reloj?
     \end{enumerate}


     % 7 
     \question%
     Considera el sistema lineal $\dfrac{d\mathbf{Y}}{dt}=\colvec{-2\quad-1}{\phantom{-}1\quad-4}\mathbf{Y}$ con la condición inicial $\mathbf{Y}_0=(1,0)$.
     \begin{enumerate}[a)]
         \item Encuentra el eigenvalor.
         \item Encuentra un eigenvector.
         \item Esboza el campo de direcciones.
         \item Bosqueja el plano fase, incluyendo la solución con la condición inicial dada.
         \item Encuentra la solución general.
         \item Calcula la solución particular para la condición inicial dada.
         \item Dibuja las gráficas $x(t)$ y $y(t)$ de la solución con la condición inicial dada.
     \end{enumerate}


     
     
     % 8 
     \question%
     El sistema lineal $\dfrac{d\mathbf{Y}}{dt}=\colvec{4\quad2}{2\quad1}\mathbf{Y}$ tiene cero como un eigenvalor.
     \begin{enumerate}[a)]
         \item Encuentra los eigenvalores.
         \item Encuentra los eigenvectores.
         \item Esboza el plano fase.
         \item Determina la solución general.
         \item Encuentra la solución particular para la condición incial $\mathbf{Y}_0=(1,0)$.
     \end{enumerate}


   
     
        \end{questions}

    
    


 
 

 

\end{document}
